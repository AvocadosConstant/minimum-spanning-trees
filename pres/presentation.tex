\documentclass{beamer}
\usepackage{apacite}
\usepackage{filecontents}
\usepackage{graphicx}

\title{Minimum Spanning Tree Algorithms}
\subtitle{CS 375 Final Project}
\author{Tim Hung and Samuel David Bravo}
\institute{Binghamton University}
\date{May 3, 2016}

\usetheme{Copenhagen}
\setbeamertemplate{itemize items}[default]
\setbeamertemplate{enumerate items}[default]
\setbeamertemplate{navigation symbols}{}%remove navigation symbols
\setbeamertemplate{section page}
{
  \begin{centering}
    \begin{beamercolorbox}[sep=12pt,center]{part title}
      \usebeamerfont{section title}\insertsection\par
    \end{beamercolorbox}
  \end{centering}
}
\setbeamertemplate{caption}{\raggedright\insertcaption\par}

\begin{document}
\frame{\titlepage}
\section{Overview}\frame{\sectionpage}

\subsection{Minimum Spanning Trees}
\begin{frame}{Minimum Spanning Trees}
    A minimum spanning tree connects all the vertices in a graph together into
    a tree with the lightest weight possible.
\end{frame}

\subsection{Approach}
\begin{frame}{Approach}
    Algorithms:

        \begin{itemize}
        \item Kruskal's
        \item Prim's
        \end{itemize}

    Implementations:

        \begin{itemize}
        \item Adjacency List
        \item Adjacency Matrix
        \end{itemize}
\end{frame}

\subsection{Problem Statement}
\begin{frame}{Problem Statement}
    \begin{center}
    How do Prim's and Kruskal's algorithms handle graphs of different
    densities?\\
    \bigskip
    Do they depend on whether the graph is implemented as an 
    adjacency list or an adjacency matrix?
    \end{center}
\end{frame}


\section{Prim's Algorithm}\frame{\sectionpage}
\subsection{Algorithm}
\begin{frame}{Main Idea}
    \begin{center}
    Expand the tree by adding the lightest connecting edge.
    \end{center}
\end{frame}

\begin{frame}{Pseudocode}
    extemely neat and clear fits in one slide pseudocode.
\end{frame}

\subsection{Implementation}
\begin{frame}{Implementation Details}
\end{frame}

\subsection{Analysis}
\begin{frame}{Analysis of Prim's}
    interesting features, time complexity, why?
\end{frame}


\section{Kruskal's Algorithm}\frame{\sectionpage}
\subsection{Algorithm}
\begin{frame}{Main Idea}
    \begin{itemize}
    \item Separate vertices into disjoint sets
    \item Reorder all edges by smallest weight first
    \item Loop through edges, add it to tree if its vertices are disjoint
    \end{itemize}
\end{frame}


\defverbatim[colored]\vkruskal {
\begin{verbatim}
EdgeContainer all_edges = graph.sorted_edges()

VerticesSet set = disjoint_set(v.size)
EdgeContainer MST = empty

for (Edge e : all_edges)
    v1 = e.source;
    v2 = e.destination
    if (set.are_vertices_disjoint(v1,v2))
        mst += e;
        set.join(v1,v2)
\end{verbatim}
}
\begin{frame}{Pseudocode}
\vkruskal
\end{frame}

\subsection{Implementation}
\begin{frame}{Key Data Structures}
    \begin{itemize}
    \item A vector of all the sorted edges
    \item A vector for holding edges included to minimum spanning tree
    \item A vector representing disjoint sets
    \end{itemize}
\end{frame}

\begin{frame}{Functions}
    \begin{itemize}
    \item ReadFromAdjacencyList or ReadFromAdjacencyMatrix
    \item SortEdgesSmallestFirst, size $|E|\log||E|$
    \item CreateDisjointSets, size $|V|$
    \item JoinSets, size $|V|$
    \item AreSetsDisjoint, size $O(1)$
    \end{itemize}
\end{frame}

\subsection{Analysis}
\begin{frame}{Analysis of Kruskal's}
    \begin{itemize}
    \item Only uses graph representation for retrieving edges
    \item Sorting the edges is of size $|E|\log|E|$
    \item Meanwhile, looping through edges is of size $|E|$
    \item The time complexity is carried by the sort, $|E|\log|E|$
    \item Kruskal's works well with sparse graphs
    \end{itemize}   
\end{frame}


\section{Results}\frame{\sectionpage}
\subsection{Demo}
\begin{frame}{Demonstration}
\end{frame}

\subsection{Data Set}
\begin{frame}{Our Data}
    Describe the dataset that you used to test the algorithm. How did you
    generate it? What characteristics does it have, and why? What did you
    decide to vary in the input set, and why?
\end{frame}

\subsection{Results}
\begin{frame}{Results}
    What did you learn from testing your algorithm?
\end{frame}

\subsection{Limitations}
\begin{frame}{Limitations and Future Work}
    What limitations does your project currently exhibit? If you had another
    month, what could you improve? What additional tests would you run?

    \begin{itemize}
    \item Kruskal's can finish early by checking if there is only 1 set. If
    that's the case, the for loop will finish in $|V|$ instead of $|E|$.
    \end{itemize}
\end{frame}


\section{Summary}\frame{\sectionpage}
\begin{frame}{Recap}
    This is a recap of what we have talked about.
\end{frame}

\begin{frame}{Questions}
    Thank you.\\
    Any questions?
\end{frame}

\end{document}
\grid
